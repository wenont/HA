\documentclass[a4paper,11pt]{scrartcl}
\usepackage[a4paper, left=2cm, right=4.5cm, top=2cm, bottom=2cm]{geometry} % kleinere Ränder

%Paket für Header in Koma-Klassen (scrartcl, scrrprt, scrbook, scrlttr2)
\usepackage[headsepline]{scrlayer-scrpage}
% Header groß genug für 3 Zeilen machen
\setlength{\headheight}{3\baselineskip}


% Default Header löschen
\pagestyle{scrheadings}
\clearpairofpagestyles

% nicht kursiv gedruckte header
\setkomafont{pagehead}{\sffamily\upshape}

% Links im Dokuement sowie \url schön machen
\usepackage[colorlinks,pdfpagelabels,pdfstartview = FitH, bookmarksopen = true,bookmarksnumbered = true, linkcolor = black, plainpages = false, hypertexnames = false, citecolor = black]{hyperref}

% Umlaute in der Datei erlauben, auf deutsch umstellen
\usepackage[utf8]{inputenc}
\usepackage[ngerman]{babel}

% Mathesymbole und Ähnliches
\usepackage{amsmath}
\usepackage{mathtools}
\usepackage{amssymb}
\usepackage{microtype}


% Komplexitätsklassen
\newcommand{\pc}{\ensuremath{{\sf P}}}
\newcommand{\np}{\ensuremath{{\sf NP}}}
\newcommand{\npc}{\ensuremath{{\sf NPC}}}
\newcommand{\pspace}{\ensuremath{{\sf PSPACE}}}
\newcommand{\exptime}{\ensuremath{{\sf EXPTIME}}}
\newcommand{\CClassNP}{\textup{NP}\xspace}
\newcommand{\CClassP}{\textup{P}\xspace}

% Weitere pakete
\usepackage{multicol}
\usepackage{booktabs}

% Abbildungen
\usepackage{tikz}
\usetikzlibrary{arrows,calc}





% Meistens ist \varphi schöner als \phi, genauso bei \theta
\renewcommand{\phi}{\varphi}
\renewcommand{\theta}{\vartheta}

% Aufzählungen anpassen (alternativ: \arabic, \alph)
\renewcommand{\labelenumi}{(\roman{enumi})}


\usepackage{color}
\usepackage{pifont}

% rwth colors
% colors: blue violet purple carmine red magenta orange yellow grass cyan gold silver
\definecolor{rwth-blue}{cmyk}{1,.5,0,0}\colorlet{rwth-lblue}{rwth-blue!50}\colorlet{rwth-llblue}{rwth-blue!25}
\definecolor{rwth-violet}{cmyk}{.6,.6,0,0}\colorlet{rwth-lviolet}{rwth-violet!50}\colorlet{rwth-llviolet}{rwth-violet!25}
\definecolor{rwth-purple}{cmyk}{.7,1,.35,.15}\colorlet{rwth-lpurple}{rwth-purple!50}\colorlet{rwth-llpurple}{rwth-purple!25}
\definecolor{rwth-carmine}{cmyk}{.25,1,.7,.2}\colorlet{rwth-lcarmine}{rwth-carmine!50}\colorlet{rwth-llcarmine}{rwth-carmine!25}
\definecolor{rwth-red}{cmyk}{.15,1,1,0}\colorlet{rwth-lred}{rwth-red!50}\colorlet{rwth-llred}{rwth-red!25}
\definecolor{rwth-magenta}{cmyk}{0,1,.25,0}\colorlet{rwth-lmagenta}{rwth-magenta!50}\colorlet{rwth-llmagenta}{rwth-magenta!25}
\definecolor{rwth-orange}{cmyk}{0,.4,1,0}\colorlet{rwth-lorange}{rwth-orange!50}\colorlet{rwth-llorange}{rwth-orange!25}
\definecolor{rwth-yellow}{cmyk}{0,0,1,0}\colorlet{rwth-lyellow}{rwth-yellow!50}\colorlet{rwth-llyellow}{rwth-yellow!25}
\definecolor{rwth-grass}{cmyk}{.35,0,1,0}\colorlet{rwth-lgrass}{rwth-grass!50}\colorlet{rwth-llgrass}{rwth-grass!25}
\definecolor{rwth-green}{cmyk}{.7,0,1,0}\colorlet{rwth-lgreen}{rwth-green!50}\colorlet{rwth-llgreen}{rwth-green!25}
\definecolor{rwth-cyan}{cmyk}{1,0,.4,0}\colorlet{rwth-lcyan}{rwth-cyan!50}\colorlet{rwth-llcyan}{rwth-cyan!25}
\definecolor{rwth-teal}{cmyk}{1,.3,.5,.3}\colorlet{rwth-lteal}{rwth-teal!50}\colorlet{rwth-llteal}{rwth-teal!25}
\definecolor{rwth-gold}{cmyk}{.35,.46,.7,.35}
\definecolor{rwth-silver}{cmyk}{.39,.31,.32,.14}

% 请把新添加的宏包放置此处
\usepackage{blindtext} 
\usepackage{enumitem}
%\usepackage {ctex} % 中文字体支持
\usepackage[most]{tcolorbox} 
% https://tex.stackexchange.com/questions/180325/boxed-equation-with-number/180326
% http://mirrors.dotsrc.org/ctan/macros/latex/contrib/mathtools/empheq.pdf#page=23

\tcbset{colback=rwth-blue!10!white, colframe=-rwth-red!50!black, 
	highlight math style= {enhanced, %<-- needed for the ’remember’ options
		colframe=red,colback=red!10!white,boxsep=0pt}
} 
%   for tcolorbox
% 	Example
%	\begin{tcolorbox}[ams gather*]
%		\sum\limits_{n=1}^{\infty} \frac{1}{n} = \infty.\\
%		\int x^2 ~\text{d}x = \frac13 x^3 + c.
%	\end{tcolorbox}

% Codeblock and Code
\usepackage{listings}
\usepackage{xcolor}

\definecolor{codegreen}{rgb}{0,0.6,0}
\definecolor{codegray}{rgb}{0.5,0.5,0.5}
\definecolor{codepurple}{rgb}{0.58,0,0.82}
\definecolor{backcolour}{rgb}{0.95,0.95,0.92}

\lstdefinestyle{mystyle}{
	%	backgroundcolor=\color{backcolour},   
	backgroundcolor=\color{white},   
	commentstyle=\color{codegreen},
	keywordstyle=\color{magenta},
	numberstyle=\tiny\color{codegray},
	stringstyle=\color{codepurple},
	basicstyle=\ttfamily\footnotesize,
	breakatwhitespace=false,         
	breaklines=true,                 
	captionpos=b,                    
	keepspaces=true,                 
	numbers=left,                    
	numbersep=5pt,                  
	showspaces=false,                
	showstringspaces=false,
	showtabs=false,                  
	tabsize=2
}

\lstset{style=mystyle}
% https://upload.wikimedia.org/wikipedia/commons/2/2d/LaTeX.pdf#page=131

% Header i-> inner (bei einseitig links), c -> center, o -> Outer (bei einseitg rechts)
\ihead{BuK WS 2020/21 \\ Tutorium 08 \\\today}
\chead{\Large Übungsblatt 05}
\ohead{Jiaming Yao, 416649 \\
	   Xiaoting Wang, 406267 \\
	   Wensheng Zhang, 405521}	
\cfoot*{\pagemark} % Seitenzahlen unten

\begin{document}
	
\section*{Aufgabe 4}
Zu zeigen: $A$ ist entscheidbar \\
$A \le B$ und $B$ rekursiv aufz{\"a}hlbar $\Longrightarrow$ $A$ ist auch rekursiv aufz{\"a}hlbar (Lemma von Reduktionen) \textcolor{blue}{- - - - \ding{172}}\\
$A \le B$ und $B \le \overline{A}$ $\Longrightarrow$ $A \le \overline{A}$ (nach Tutoraufgabe 2a von 4.Tutuorium, also Transitivit{\"a}t von Reduktionen) \\
$A \le \overline{A}$ $\Longrightarrow$ $\overline{A} \le A$ (nach Tutoraufgabe 2b von 4.Tutuorium) \textcolor{blue}{- - - - \ding{173}}\\
\textcolor{blue}{\ding{172}} und \textcolor{blue}{ \ding{173}} $\Longrightarrow$ $ \overline{A}$ ist auch rekursiv aufz{\"a}hlbar  \textcolor{blue}{- - - -  \ding{174}}\\
\textcolor{blue}{\ding{172}} und \textcolor{blue}{\ding{174}} $\Longrightarrow$ $A$ ist entscheidbar 


 
\section*{Aufgabe 5}
\noindent Zu zeigen: $\textbf{H}_{\varepsilon}$ $\le$ $\textbf{L}_{111}$ \\
\noindent Sei $\omega$ die Eingabe f{\"u}r $\textbf{H}_{\varepsilon}$
\begin{itemize}
	\item Wenn $\omega$ keine g{\"u}ltige G{\"o}delnummer ist, so sei $f(\omega)=\omega$
	\item Falls $\omega=\langle M \rangle$ f"ur eine \textbf{TM} M, so sei $f(\omega)$ die G"odelnummer einer \text{TM} $M^{*}$ mit der folgenden Eigenschaften:\\
	$M^{*}$ "uberpr"uft, ob die Eingabe mit $111$ endet.\\
	\underline{Falls ja}, l"oscht $M^{*}$ die Eingabe und simuliert $M$ mit der Eingabe $\varepsilon$. Ansonsten geht $M^*$ in eine Endlosschleife.
\end{itemize} 

\noindent Korrektheit:
\begin{itemize}
	\item Falls $\omega$ keine G"odelnummer ist, ist die Korrektheit klar
	\item Sei nun $\omega=\langle M \rangle$ f"ur eine \textbf{TM} und sei $f(\omega)=\langle M^{*} \rangle$\\
	Es gilt: $\omega \in \textbf{H}_{\varepsilon}$\\
	$\Longrightarrow$ M h"alt auf der Eingabe $\varepsilon$\\
	$\Longrightarrow$ $M^*$ h"alt auf der Eingabe, die mit $111$ endet.\\
	$\Longrightarrow$ $\langle M^* \rangle \in \textbf{L}_{111}$\\
	$\Longrightarrow$ $f(\omega) \in \textbf{L}_{111}$\\
	
	
	$\omega \notin \textbf{H}_{\varepsilon}$ \\
	$\Longrightarrow$ M h"alt nicht auf der Eingabe $\varepsilon$\\    
	$\Longrightarrow$ $M^*$ h"alt nicht auf der Eingabe, die mit $111$ enden\\
	$\Longrightarrow$ $\langle M^* \rangle \notin \textbf{L}_{111}$\\
	$\Longrightarrow$ $f(\omega) \notin \textbf{L}_{111}$\\
	
\end{itemize}


\section*{Aufgabe 6}

\subsection*{a)}

$L_{\mathbb{P}}$ ist unentscheidbar.\\
Wir beweisen es durch \textit{Satz von Rice}.\\
$S =\{ f_M\ |\ f_M(\mathbb{P}) = 1,\ f_M(\Sigma^*\backslash \mathbb{P})=0\}$
\begin{align*}
	L_{\mathbb{P}} &= L(S)\\
	&= \{\langle M\rangle \ |\ M\ berechnet\ eine\ Funktion\ aus\ S\}\\
	&= \{\langle M\rangle \ |\ M\  entscheidet\ die\ Menge\ der\ Bin"ardarstellugungen\ der\ Primzahlen. \}
\end{align*}
\begin{itemize}
	\item 
	$S \ne \emptyset :$\\
	Es existiert eine TM $M_{10}$ mit:\\
	$M_{10}$ kann 2 (deren Bin"ardarstellung ist (10) entscheiden. D.h. $M_{10}$ akzeptiert 10. Ansonsten verwirft $M_{10}$.\\
	$f_{M_{10}} \in S \Longrightarrow S \ne \emptyset$
	
	\item
	$S \ne R$\\
	Es existiert so eine TM $M_{\neg (10)}$ mit:\\
	$M_{\neg (10)}$ kann auch 2 (deren Bin"ardarstellung ist 10) entscheiden. Aber im Fall verwirft $M_{\neg (10)}$ 10. Ansonsten akzeptiert $M_{\neg (10)}$ immer.\\
	$f_{M_{\neg (10)}} \in R\backslash S \Longrightarrow S \ne R$\\
	
	Nach \textit{Satz von Rice} ist $L_{\mathbb{P}}$ unentscheidbar.
		
\end{itemize}

\subsection*{b)}
	Wir definieren $L_{comp} = \{ \langle M_1 \rangle\ \langle M_2 \rangle \ |\ L(M_1) = \overline{L(M_2)}\}$
	\\ \ \\
	Zu Zeigen: $H_\epsilon \leq L_{comp}$\\
	Beschreibung der Funktion f:\\
	Sei w die Eingabe f"ur $H_\epsilon$.
	\begin{itemize}
		\item Wenn $w$ keine G"odelnummer ist, so sei $f(w) = w$.
		\item Falls $w = \langle M \rangle$ f"ur ein TM $M$, so sei $f(w)$ die G"odelnummer von TM $M^*_1$ und $M^*_2$, die die folgenden Eigenschaft haben:
		 \begin{itemize}
		 	\item[$M^*_1$] l"osche die Eingabe und simuliert M auf $\epsilon$. Falls M in den Endzustand l"auft(M h"alt), \textbf{schreibt $M^*_1$ ein 1 auf dem Band}.
		 	\item[$M^*_2$] l"osche die Eingabe und simuliert M auf $\epsilon$. Falls M in den Endzustand l"auft(M h"alt), dann \textbf{geht $M^*_2$ in eine Endlosschleife}.
		 \end{itemize}

	\end{itemize}

	Korrektheit:
	\begin{align*}
		w\in H_\epsilon &\Longrightarrow M\ h"alt\ auf\ \epsilon \\
		&\Longrightarrow M^*_1\ akzeptiert\ die\ Einegabe.\ M^*_2\ akzeptiert\ dieselbe\ Eingabe\ nicht.\\
		&\Longrightarrow \langle M^*_1 \rangle\ \langle M^*_2 \rangle \in L_{comp}\\
		&\Longrightarrow f(w) \in L_{comp}\\		
		w\notin H_\epsilon &\Longrightarrow M\ h"alt\ nicht\ auf\ \epsilon \\
		&\Longrightarrow  M^*_1\ akzeptiert\ alle\ Einegabe nicht.\ M^*_2\ akzeptiert\ alle\ Eingabe\ nicht.\\
		&\Longrightarrow \langle M^*_1 \rangle\ \langle M^*_2 \rangle \notin L_{comp}\\
		&\Longrightarrow f(w) \notin L_{comp}\\		
	\end{align*}

	Daher wird $H_\epsilon \leq L_{comp}$ zeigt. Da $H_\epsilon$ nicht rekursiv ist, ist $L_{comp}$ nicht rekursiv.
	
	
\section*{Aufgabe 7}
\subsection*{a)}
	Zu zeigen:\\
	$L$ ist rekursiv aufz{\"a}hlbar $\iff$ $L=$ Def($f$)=\{$x$ $|$ $f(x) \ne \bot$ \}
	\begin{itemize}		
		\item[$``\Rightarrow "$] Sei A ein Aufz{\"a}hler f{\"u}r $L$. Wir konstruieren eine TM $M$, die $L$ erkennt.\\
		\textcolor{blue} {Bei Eingabe $w$ arbeitet $M$ wir folgt:}\\
		$M$ simuliert $A$ mit Hilfe einer Spur, welche die Rolle des Druckers {\"u}bernimmt.\\
		Immer wenn ein neues Wort gedruckt worden ist, vergleicht $M$ dieses Wort mit $w$ und h{\"a}lt bei {\"U}bereinstimmung auf.\\
		Daher berechnet TM $M$ die Funktion $f_M$ mit der Form: $\forall$ $x$ $\in$ $L$, $f_M(x) \ne \bot$
		
		\item[$``\Leftarrow "$] 
		Sei $f$ eine berechenbare Funktion f"ur $S=\{f\ |\ \forall x\in L,\ f(x)\ne \bot\}$\\
		F"ur die Funktion f existiert eine TM M', die f berechnet.\\
		Dann konstruieren wir ein Aufz"ahler $A'$ durch $M'$:\\
		\textcolor{blue}{F"ur $i = 1, 2, 3\cdots$ }\\
		$A'$ simuliere je i Schritte von $M'$ auf jedem Wort aus $\{w_1,\ w_2,\cdots ,w_i\}$.\\
		Wann $M'$ immer dabei auf eines der W"orter h"alt, so drucke es aus.\\
		Somit ist $A'$ ein Aufz"ahler von L.
	\end{itemize}

\subsection*{b)}

	Zu zeigen:\\
	$L$ ist rekursiv aufz{\"a}hlbar $\iff$ $L=$ Bild($f$)=\{$f(x)$ $|$ $x\in \{0,1\}^*$ \} oder $L$=\{\}\\	
	\begin{itemize}		
		\item[$``\Rightarrow "$] 
		Sei $L$ rekursiv aufz"ahlbar. Damit gibt es F"ur $L$ ein Aufz"ahler A.\\
		F"ur $\{0,1\}^*$ steht eine kanonische Reihenfolge. Sei $x\in \{0,1\}^*$ in i-Position
		
		Konstruiere ein TM M mit folgenden Eigenschaft: 
		\begin{itemize}
			\item [1.] Suche, welche Stelle die Eingabe in der kanonischen Reihenfolge steht, z.B. i-te Stelle.
			\item [2.] Simuliere A und druck das i-te Wort von L aus. Falls A h"alt(D.h. A hat alle W"orter von L ausgegeben), dann simuliere A nochmal, aber z"ahle weiter. (Z.B. Das letzte Wort aus A ist n-te Wort. Wenn wir nochmal A simulieren, z"ahlen wir von n+1.)
			\item [3.] Wiederhole 2-te Schritte, bis das i-te Wort von $L$ ausgibt.
		\end{itemize}
	Im Fall kann $M$ f"ur jeder $x\in \{0,1\}^*$ ein Wort von L finden und dann es ausgeben. F"ur so eine TM $M$ existiert ein Funktion, die total berechenbar ist.
		
		

	\end{itemize}

	 
\end{document}

