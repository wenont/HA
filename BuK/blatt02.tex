\documentclass[a4paper,11pt]{scrartcl}
\usepackage[a4paper, left=2cm, right=4.5cm, top=2cm, bottom=2cm]{geometry} % kleinere Ränder

%Paket für Header in Koma-Klassen (scrartcl, scrrprt, scrbook, scrlttr2)
\usepackage[headsepline]{scrlayer-scrpage}
% Header groß genug für 3 Zeilen machen
\setlength{\headheight}{3\baselineskip}


% Default Header löschen
\pagestyle{scrheadings}
\clearpairofpagestyles

% nicht kursiv gedruckte header
\setkomafont{pagehead}{\sffamily\upshape}

% Links im Dokuement sowie \url schön machen
\usepackage[colorlinks,pdfpagelabels,pdfstartview = FitH, bookmarksopen = true,bookmarksnumbered = true, linkcolor = black, plainpages = false, hypertexnames = false, citecolor = black]{hyperref}

% Umlaute in der Datei erlauben, auf deutsch umstellen
\usepackage[utf8]{inputenc}
\usepackage[ngerman]{babel}

% Mathesymbole und Ähnliches
\usepackage{amsmath}
\usepackage{mathtools}
\usepackage{amssymb}
\usepackage{microtype}
\newcommand{\NN}{{\mathbb N}}
\newcommand{\RR}{{\mathbb R}}
\newcommand{\QQ}{{\mathbb Q}}
\newcommand{\ZZ}{{\mathbb{Z}}}

% Komplexitätsklassen
\newcommand{\pc}{\ensuremath{{\sf P}}}
\newcommand{\np}{\ensuremath{{\sf NP}}}
\newcommand{\npc}{\ensuremath{{\sf NPC}}}
\newcommand{\pspace}{\ensuremath{{\sf PSPACE}}}
\newcommand{\exptime}{\ensuremath{{\sf EXPTIME}}}
\newcommand{\CClassNP}{\textup{NP}\xspace}
\newcommand{\CClassP}{\textup{P}\xspace}

% Weitere pakete
\usepackage{multicol}
\usepackage{booktabs}

% Abbildungen
\usepackage{tikz}
\usetikzlibrary{arrows,calc}





% Meistens ist \varphi schöner als \phi, genauso bei \theta
\renewcommand{\phi}{\varphi}
\renewcommand{\theta}{\vartheta}

% Aufzählungen anpassen (alternativ: \arabic, \alph)
\renewcommand{\labelenumi}{(\roman{enumi})}

% rwth colors
% colors: blue violet purple carmine red magenta orange yellow grass cyan gold silver
\definecolor{rwth-blue}{cmyk}{1,.5,0,0}\colorlet{rwth-lblue}{rwth-blue!50}\colorlet{rwth-llblue}{rwth-blue!25}
\definecolor{rwth-violet}{cmyk}{.6,.6,0,0}\colorlet{rwth-lviolet}{rwth-violet!50}\colorlet{rwth-llviolet}{rwth-violet!25}
\definecolor{rwth-purple}{cmyk}{.7,1,.35,.15}\colorlet{rwth-lpurple}{rwth-purple!50}\colorlet{rwth-llpurple}{rwth-purple!25}
\definecolor{rwth-carmine}{cmyk}{.25,1,.7,.2}\colorlet{rwth-lcarmine}{rwth-carmine!50}\colorlet{rwth-llcarmine}{rwth-carmine!25}
\definecolor{rwth-red}{cmyk}{.15,1,1,0}\colorlet{rwth-lred}{rwth-red!50}\colorlet{rwth-llred}{rwth-red!25}
\definecolor{rwth-magenta}{cmyk}{0,1,.25,0}\colorlet{rwth-lmagenta}{rwth-magenta!50}\colorlet{rwth-llmagenta}{rwth-magenta!25}
\definecolor{rwth-orange}{cmyk}{0,.4,1,0}\colorlet{rwth-lorange}{rwth-orange!50}\colorlet{rwth-llorange}{rwth-orange!25}
\definecolor{rwth-yellow}{cmyk}{0,0,1,0}\colorlet{rwth-lyellow}{rwth-yellow!50}\colorlet{rwth-llyellow}{rwth-yellow!25}
\definecolor{rwth-grass}{cmyk}{.35,0,1,0}\colorlet{rwth-lgrass}{rwth-grass!50}\colorlet{rwth-llgrass}{rwth-grass!25}
\definecolor{rwth-green}{cmyk}{.7,0,1,0}\colorlet{rwth-lgreen}{rwth-green!50}\colorlet{rwth-llgreen}{rwth-green!25}
\definecolor{rwth-cyan}{cmyk}{1,0,.4,0}\colorlet{rwth-lcyan}{rwth-cyan!50}\colorlet{rwth-llcyan}{rwth-cyan!25}
\definecolor{rwth-teal}{cmyk}{1,.3,.5,.3}\colorlet{rwth-lteal}{rwth-teal!50}\colorlet{rwth-llteal}{rwth-teal!25}
\definecolor{rwth-gold}{cmyk}{.35,.46,.7,.35}
\definecolor{rwth-silver}{cmyk}{.39,.31,.32,.14}

% 请把新添加的宏包放置此处
\usepackage{blindtext} 
\usepackage{enumitem}
%\usepackage {ctex} % 中文字体支持
\usepackage[most]{tcolorbox} 
% https://tex.stackexchange.com/questions/180325/boxed-equation-with-number/180326
% http://mirrors.dotsrc.org/ctan/macros/latex/contrib/mathtools/empheq.pdf#page=23

\tcbset{colback=rwth-blue!10!white, colframe=-rwth-red!50!black, 
	highlight math style= {enhanced, %<-- needed for the ’remember’ options
		colframe=red,colback=red!10!white,boxsep=0pt}
} 
%   for tcolorbox
% 	Example
%	\begin{tcolorbox}[ams gather*]
%		\sum\limits_{n=1}^{\infty} \frac{1}{n} = \infty.\\
%		\int x^2 ~\text{d}x = \frac13 x^3 + c.
%	\end{tcolorbox}

\usepackage[ampersand]{easylist}
\ListProperties(Hide=100, Hang=true, Progressive=3ex, Style*=-- ,Style2*=$\bullet$ ,Style3*=$\circ$ ,Style4*=\tiny$\blacksquare$ ) %easylist
% ...
% 	example:
%	\begin{easylist}
%		& Blah
%		& Blah
%		&& Blah
%		&&& Blah
%		&&&& Blah
%		&&&&& Blah
%	\end{easylist}
% https://upload.wikimedia.org/wikipedia/commons/2/2d/LaTeX.pdf#page=131

% Header i-> inner (bei einseitig links), c -> center, o -> Outer (bei einseitg rechts)
\ihead{BuK WS 2020/21 \\ Tutorium 08 \\\today}
\chead{\Large Übungsblatt 02}
\ohead{Jiaming Yao, 416649 \\
	   Xiaoting Wang, 406267 \\
	   Wensheng Zhang, 405521}	
\cfoot*{\pagemark} % Seitenzahlen unten

\begin{document}

\section*{Aufgabe 5}

Schreibe das gleiche Wirt w auf 1-Band und 2-Band. Der Kopf steht "uber dem letzten Zeichen von 1-und 2-Band. \\ \ \\
Erster Schritt:\\
Bewege den Kopf von 1-Band nach Links. Pr"ufe dabei, ob die L"ange vom Wort gerade ist. Inzwischen macht 2-Band nicht.


\begin{table}[h!]
	\begin{tabular}{|l|l|l|l|}
		\hline
		& 0         & 1         & B         \\ \hline
		$q_0$ & $q_1,0,L$ & $q_1,1,L$ & $q_2,B,R$ \\ \hline
		$q_1$ & $q_0,0,L$ & $q_0,1,L$ & Reject    \\ \hline
	\end{tabular}
\end{table}


Zweiter Schritt:\\
Bewege den Kopf von 1-Band nach recht und den Kopf von 2-Band nach links. Ist gelesene Zeichen von beiden B"andern gleich, schreibe ein Blank dann gehe weiter, ansonsten reject.


\begin{table}[h!]
	\begin{tabular}{|l|l|l|l|}
		\hline
		& 00/11      & 01/10  & BB     \\ \hline
		$q_2$ & $q_2,BB,RL$ & Reject & Accept \\ \hline
	\end{tabular}
\end{table}

	
\section*{Aufgabe 7}

\subsection*{a)}

	Wir nehmen eine Aussage an, dass R abz"ahlbar ist.\\ 

	Aus der Vorlesung haben wir gelernt, dass $\sum ^{*}$, die Menge der W"orter "uber einem endlichen Alphabet $\sum$, abz"ahlbar ist.\\
	
	D.h. die Menge von aller W"orter, die aus $\{ a,\ b,\ c\}$ besteht, ist auch abz"ahlbar.\\
	
	% $\{ a,\ b,\ c\ ,{}^{*}\ +,\ (,\ ), \emptyset \}$
	
	Sei W die Menge von allen W"orter von R und $w_1,\ w_2,\ w_3,\ w_4,\dots$ eine Aufz"ahlung von W.\\
	
	Wir definieren eine 2-dimensionale unendliche Matrix $(A_{i,j})_{i\in \mathbb{N},j\in \mathbb{N}}$ mit
	
	
	$$ A_{i,j}=\left\{
	\begin{aligned}
		1 &  & falls\ w_j \in L(r_i)\\
		0 &  & sonst\\
	\end{aligned}
	\right.
	$$
	
	Wir definieren die Menge
	$$
	W_{diag} = \{ w_i\ |\ i\in \mathbb{N},\ A_{i,i} = 1 \} 
	$$
	
	$\Longrightarrow$ das Komplement von $W_{diag}$:
	
	$$
	\bar{W}_{diag} = W\backslash W_{diag} = \{ w_i\ |\ i\in \mathbb{N},\ A_{i,i} = 0 \} 
	$$
	
	$\Longrightarrow$ $\bar{W}_{diag} \in W$\\
	
	Sei $w_{l_1},\ w_{l_2},\ w_{l_3},\dots$ eine Abz"ahlung von $\bar{W}_{diag}$ mit einem regul"aren Ausdruck $r_k = (r_{m_1} + r_{m_2} + r_{m_3} + \dots )$ f"ur $k\in \mathbb{N}$, $w_{L_1}\in L(r_{m_1})$, $w_{l_2} \in L(r_{m_2})$, usw., denn R und W beide sind abz"ahlbar.\\ 
	
	Somit $\bar{W}_{diag} =\{ w_{l_1},\ w_{l_2},\ w_{l_3},\dots \} = L(r_{m_1}) \cup L(r_{m_2}) \cup L(r_{m_3}) \cup \dots = (r_{m_1} + r_{m_2} + r_{m_3} + \dots ) = L(r_k)$. Dazu gibt es auch ein entsprechendes Wort $w_k$.\\
	
	Jetzt betrachten wir zwei F"alle:
	
	\begin{itemize}
		\item Fall 1
		$$
		A_{k,k} = 0 \overset{Def\ \bar{W}_{diag}}{\Longrightarrow} w_k \notin \bar{W}_{diag} \Longrightarrow w_k \notin L(r_k)	 \overset{Def\ A}{\Longrightarrow} A_{k,k} = 1
		$$
		
		\item Fall 2
		$$
		A_{k,k} = 1 \overset{Def\ \bar{W}_{diag}}{\Longrightarrow} w_k \in \bar{W}_{diag} \Longrightarrow w_k \in L(r_k) \overset{Def\ A}{\Longrightarrow} A_{k,k} = 0
		$$		
	\end{itemize}
	
	Beide F"alle zeigen den Widerspruch. Somit ist R "uberz"ahlbar.
	
	





















	% Hier geht die eigentliche Lösung der Aufgaben los
	
%	\section*{Aufgabe 1}
%	
%	Das \LaTeX{}-Kompendium auf Wikibooks\footnote{\url{https://de.wikibooks.org/wiki/LaTeX-Kompendium}} ist eine gute Einführung.
%	Für die Zusammenarbeit bietet sich das RWTH Gitlab an\footnote{\url{https://git.rwth-aachen.de}}.
%	
%	
%	\section*{Aufgabe 2}
%	
%	Als \LaTeX{}-Editor bzw. -IDE ist TeXStudio \footnote{\url{https://www.texstudio.org/}} ein guter Kandidat.
%	
%
%	
%	\section*{Aufgabe 3}
%	
%		Für Teilaufgaben kann man Aufzählungen verwenden:
%	\begin{itemize}
%		\item als Aufzählung
%		\item ohne Nummerierung,
%	\end{itemize}
%	oder auch
%	\begin{enumerate}
%		\item mit
%		\item Nummerierung.
%	\end{enumerate}
%	
%	\section*{Aufgabe 4}
%	
%	Kleine Diagramme lassen sich ebenfalls erstellen, zum Beispiel:
%	
%	\begin{center}
%		\begin{tikzpicture}
%		\node (1) {$v_1$};
%		\node [right of=1] (2) {$v_2$};
%		\node [below of=2] (3) {$v_3$};
%		\draw (1) -- (2);
%		\draw (1) -- (3);
%		\end{tikzpicture}
%	\end{center}
%	
%	oder auch etwas komplizierter:
%	
%	\begin{center}
%		% Stil festlegen, der dann für alle Knoten verwendet wird
%		\tikzset{dot/.style={circle, draw=black, fill=black, inner sep=0pt, minimum size=5pt}}
%		% Das eigentlich Diagramm
%		\begin{tikzpicture}[node distance=1.5cm]
%		\node [dot,label={left:$v$}] (1) {};
%		\node [dot,right of=1] (2) {};
%		\node [dot,below of=2,label={right:$w$}] (3) {};
%		\node [dot,below of=1] (4) {};
%		\path [draw=black, ->, >=stealth', shorten <=2pt, shorten >=2pt] % Optionen nur für das Aussehen
%		(1) edge (3)
%		(1) edge (4)
%		(2) edge (3)
%		(4) edge [loop left] ()
%		(1) edge node [above] {$e$} (2);
%		\end{tikzpicture}
%	\end{center}
%	
%	Alternativ können auch externe Dateien eingebunden werden (z.B.\ Bildformate, PDF).
%	% Dazu ganz oben das Paket graphicx einbinden: \usepackage{graphicx}
%	% und an der entsprechenden Stelle dann die Grafik laden: \includegraphics{datei}
%	
\end{document}

