\documentclass[a4paper,11pt]{scrartcl}
\usepackage[a4paper, left=2cm, right=4.5cm, top=2cm, bottom=2cm]{geometry} % kleinere Ränder

%Paket für Header in Koma-Klassen (scrartcl, scrrprt, scrbook, scrlttr2)
\usepackage[headsepline]{scrlayer-scrpage}
% Header groß genug für 3 Zeilen machen
\setlength{\headheight}{3\baselineskip}


% Default Header löschen
\pagestyle{scrheadings}
\clearpairofpagestyles

% nicht kursiv gedruckte header
\setkomafont{pagehead}{\sffamily\upshape}

% Links im Dokuement sowie \url schön machen
\usepackage[colorlinks,pdfpagelabels,pdfstartview = FitH, bookmarksopen = true,bookmarksnumbered = true, linkcolor = black, plainpages = false, hypertexnames = false, citecolor = black]{hyperref}

% Umlaute in der Datei erlauben, auf deutsch umstellen
\usepackage[utf8]{inputenc}
\usepackage[ngerman]{babel}

% Mathesymbole und Ähnliches
\usepackage{amsmath}
\usepackage{mathtools}
\usepackage{amssymb}
\usepackage{microtype}


% Komplexitätsklassen
\newcommand{\pc}{\ensuremath{{\sf P}}}
\newcommand{\np}{\ensuremath{{\sf NP}}}
\newcommand{\npc}{\ensuremath{{\sf NPC}}}
\newcommand{\pspace}{\ensuremath{{\sf PSPACE}}}
\newcommand{\exptime}{\ensuremath{{\sf EXPTIME}}}
\newcommand{\CClassNP}{\textup{NP}\xspace}
\newcommand{\CClassP}{\textup{P}\xspace}

% Weitere pakete
\usepackage{multicol}
\usepackage{booktabs}

% Abbildungen
\usepackage{tikz}
\usetikzlibrary{arrows,calc}





% Meistens ist \varphi schöner als \phi, genauso bei \theta
\renewcommand{\phi}{\varphi}
\renewcommand{\theta}{\vartheta}

% Aufzählungen anpassen (alternativ: \arabic, \alph)
\renewcommand{\labelenumi}{(\roman{enumi})}


\usepackage{color}
\usepackage{pifont}

% rwth colors
% colors: blue violet purple carmine red magenta orange yellow grass cyan gold silver
\definecolor{rwth-blue}{cmyk}{1,.5,0,0}\colorlet{rwth-lblue}{rwth-blue!50}\colorlet{rwth-llblue}{rwth-blue!25}
\definecolor{rwth-violet}{cmyk}{.6,.6,0,0}\colorlet{rwth-lviolet}{rwth-violet!50}\colorlet{rwth-llviolet}{rwth-violet!25}
\definecolor{rwth-purple}{cmyk}{.7,1,.35,.15}\colorlet{rwth-lpurple}{rwth-purple!50}\colorlet{rwth-llpurple}{rwth-purple!25}
\definecolor{rwth-carmine}{cmyk}{.25,1,.7,.2}\colorlet{rwth-lcarmine}{rwth-carmine!50}\colorlet{rwth-llcarmine}{rwth-carmine!25}
\definecolor{rwth-red}{cmyk}{.15,1,1,0}\colorlet{rwth-lred}{rwth-red!50}\colorlet{rwth-llred}{rwth-red!25}
\definecolor{rwth-magenta}{cmyk}{0,1,.25,0}\colorlet{rwth-lmagenta}{rwth-magenta!50}\colorlet{rwth-llmagenta}{rwth-magenta!25}
\definecolor{rwth-orange}{cmyk}{0,.4,1,0}\colorlet{rwth-lorange}{rwth-orange!50}\colorlet{rwth-llorange}{rwth-orange!25}
\definecolor{rwth-yellow}{cmyk}{0,0,1,0}\colorlet{rwth-lyellow}{rwth-yellow!50}\colorlet{rwth-llyellow}{rwth-yellow!25}
\definecolor{rwth-grass}{cmyk}{.35,0,1,0}\colorlet{rwth-lgrass}{rwth-grass!50}\colorlet{rwth-llgrass}{rwth-grass!25}
\definecolor{rwth-green}{cmyk}{.7,0,1,0}\colorlet{rwth-lgreen}{rwth-green!50}\colorlet{rwth-llgreen}{rwth-green!25}
\definecolor{rwth-cyan}{cmyk}{1,0,.4,0}\colorlet{rwth-lcyan}{rwth-cyan!50}\colorlet{rwth-llcyan}{rwth-cyan!25}
\definecolor{rwth-teal}{cmyk}{1,.3,.5,.3}\colorlet{rwth-lteal}{rwth-teal!50}\colorlet{rwth-llteal}{rwth-teal!25}
\definecolor{rwth-gold}{cmyk}{.35,.46,.7,.35}
\definecolor{rwth-silver}{cmyk}{.39,.31,.32,.14}

% 请把新添加的宏包放置此处
\usepackage{blindtext} 
\usepackage{enumitem}
%\usepackage {ctex} % 中文字体支持
\usepackage[most]{tcolorbox} 
% https://tex.stackexchange.com/questions/180325/boxed-equation-with-number/180326
% http://mirrors.dotsrc.org/ctan/macros/latex/contrib/mathtools/empheq.pdf#page=23

\tcbset{colback=rwth-blue!10!white, colframe=-rwth-red!50!black, 
	highlight math style= {enhanced, %<-- needed for the ’remember’ options
		colframe=red,colback=red!10!white,boxsep=0pt}
} 
%   for tcolorbox
% 	Example
%	\begin{tcolorbox}[ams gather*]
%		\sum\limits_{n=1}^{\infty} \frac{1}{n} = \infty.\\
%		\int x^2 ~\text{d}x = \frac13 x^3 + c.
%	\end{tcolorbox}

% Codeblock and Code
\usepackage{listings}
\usepackage{xcolor}

\definecolor{codegreen}{rgb}{0,0.6,0}
\definecolor{codegray}{rgb}{0.5,0.5,0.5}
\definecolor{codepurple}{rgb}{0.58,0,0.82}
\definecolor{backcolour}{rgb}{0.95,0.95,0.92}

\lstdefinestyle{mystyle}{
	%	backgroundcolor=\color{backcolour},   
	backgroundcolor=\color{white},   
	commentstyle=\color{codegreen},
	keywordstyle=\color{magenta},
	numberstyle=\tiny\color{codegray},
	stringstyle=\color{codepurple},
	basicstyle=\ttfamily\footnotesize,
	breakatwhitespace=false,         
	breaklines=true,                 
	captionpos=b,                    
	keepspaces=true,                 
	numbers=left,                    
	numbersep=5pt,                  
	showspaces=false,                
	showstringspaces=false,
	showtabs=false,                  
	tabsize=2
}

\lstset{style=mystyle}
% https://upload.wikimedia.org/wikipedia/commons/2/2d/LaTeX.pdf#page=131

% Header i-> inner (bei einseitig links), c -> center, o -> Outer (bei einseitg rechts)
\ihead{BuK WS 2020/21 \\ Tutorium 08 \\\today}
\chead{\Large Übungsblatt 06}
\ohead{Jiaming Yao, 416649 \\
	   Xiaoting Wang, 406267 \\
	   Wensheng Zhang, 405521}	
\cfoot*{\pagemark} % Seitenzahlen unten

\begin{document}

\section*{Aufgabe 6}

\subsection*{b)}
	Zu zeigen: $K\in PKP \iff f(K) \in PKP_b$
	\\ \ \\
	Sei \textit{K} die die Eingabe f"ur PKP. $K =\Big\{ \left[ \frac{x_1}{y_1}  \right] ,\cdots , \left[ \frac{x_k}{y_k}  \right] \Big\} $
	\begin{itemize}
		\item [Fall 1] Wenn es in \textit{K} keine Dominos gibt, wo das obere und das untere Wort gleich lang sind, so sei $f(K)=K$.
		\item [Fall 2] Ansonsten machen wir f"ur die Dominos (oben und unten gleich lang) folgendes:\\ \ \\
		Wir bezeichnen solche Dominos als $K'=\big\{  \left[ \frac{x_{l_{1}}}{y_{l_{1}}} \right], \cdots ,\left[ \frac{x_{l_{n}}}{y_{l_{n}}} \right] \big\}$. (die Reihenfolge spielt keine Rolle)\\
		Dann kombinieren wir Dominos aus $K'$ \textcolor{rwth-blue}{(oben und unten gleich lang)} mit einem anderen Domino aus $K/K'$ \textcolor{rwth-blue}{ (oben und unten verschieden lang)}:\\
		Zum Beispiel k"onnen wir  $\left[ \frac{x_{l_{1}}}{y_{l_{1}}} \right]$ mit $\left[ \frac{x_i}{y_i}  \right]$ (einer aus $K/K'$) kombinieren. In dem Fall bekommen wir zwei neue Dominos $\left[ \frac{x_{l_{1}} x_i}{y_{l_{1}} y_i} \right]$ und $\left[ \frac{x_i x_{l_{1}}}{y_i y_{l_{1}}} \right]$ \textcolor{rwth-blue}{ (einmal vorn, einmal hinter) }.
		\\ \ \\
		Insgesamt k"onnen wir $2 \times n \times (k-n)$ mal neue Dominos konstruieren. Man kann klar sehen, in solchen Dominos stehen immer verschieden lang W"orter oben und unten. Wir bezeichnen solche konstruierte Dominos als $K''$. So sei $f(K) = \{ K/K', K''\} $.
	\end{itemize}  
	
	
	\noindent Korrektheit:\\
	
	\noindent \textcolor{rwth-blue}{$K \in PKP \Longrightarrow f(K) \in PKP_b$}\\
	Sei $(i_1,\cdots ,i_n)$ eine L"osung f"ur $K$, d.h.
	\begin{align*}
		x_{i_1} x_{i_2} \cdots x_{i_n} = y_{i_1} y_{i_2} \cdots y_{i_n}
	\end{align*}
	F"ur Fall 1 ist es klar, $f(K) \in PKP_b$.\\
	Im Fall 2 gibt es immer eine L"osung f"ur $f(K)$, dessen entsprechende W"orter gleich wie oben ist.\\
	Z.B. die L"ange von $x_{i_j}$ ist gleich wie $y_{i_j}$ f"ur ein $j \in \{1,\cdots, n\}$. So ein Domino steht nicht in $f(K)$. Aber k"onnen wir dieser durch $\left[ \frac{x_{i_{j-1}} x_{i_j}}{y_{i_{j-1}} y_{i_j}} \right]$ und $\left[ \frac{x_{i_{j}} x_{i_{j+1}}}{y_{i_{j}} y_{i_{j+1}}} \right]$ aus $f(K)$ ersetzen und l"oschen die entsprechende Domino:\\
	$$
	  \left[ \frac{x_{i_{1}}}{y_{i_{1}}} \right] \cdots  \left[ \frac{x_{i_{j}}}{y_{i_{j}}} \right]  \cdots \left[ \frac{x_{i_{n}}}{y_{i_{n}}} \right] \Longrightarrow \left[ \frac{x_{i_{1}}}{y_{i_{1}}} \right] \cdots  \left[ \frac{x_{i_{j}} x_{i_{j+1}}}{y_{i_{j}} y_{i_{j+1}}} \right] \left[ \frac{x_{i_{j+2}}}{y_{i_{j+2}}} \right] \cdots \left[ \frac{x_{i_{n}}}{y_{i_{n}}} \right]
	$$
	\textcolor{rwth-blue}{$K \notin PKP \Longrightarrow f(K) \notin PKP_b$}\\
	Offensichtlich......\\
	\\ \ \\
	Wegen $K\in PKP \iff f(K) \in PKP_b$ und Unentscheidbarkeit von $PKP$ ist $PKP_b$ auch unentscheidbar.
	
	
	
	
	 
	
	




\end{document}

