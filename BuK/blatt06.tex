\documentclass[a4paper,11pt]{scrartcl}
\usepackage[a4paper, left=2cm, right=4.5cm, top=2cm, bottom=2cm]{geometry} % kleinere Ränder

%Paket für Header in Koma-Klassen (scrartcl, scrrprt, scrbook, scrlttr2)
\usepackage[headsepline]{scrlayer-scrpage}
\usepackage{color}
\usepackage{pifont}
% Header groß genug für 3 Zeilen machen
\setlength{\headheight}{3\baselineskip}


% Default Header löschen
\pagestyle{scrheadings}
\clearpairofpagestyles

% nicht kursiv gedruckte header
\setkomafont{pagehead}{\sffamily\upshape}




% Links im Dokuement sowie \url schön machen
\usepackage[colorlinks,pdfpagelabels,pdfstartview = FitH, bookmarksopen = true,bookmarksnumbered = true, linkcolor = black, plainpages = false, hypertexnames = false, citecolor = black]{hyperref}

% Umlaute in der Datei erlauben, auf deutsch umstellen
\usepackage[utf8]{inputenc}
\usepackage[ngerman]{babel}

% Mathesymbole und Ähnliches
\usepackage{amsmath}
\usepackage{mathtools}
\usepackage{amssymb}
\usepackage{microtype}
\newcommand{\NN}{{\mathbb N}}
\newcommand{\RR}{{\mathbb R}}
\newcommand{\QQ}{{\mathbb Q}}
\newcommand{\ZZ}{{\mathbb{Z}}}

% Komplexitätsklassen
\newcommand{\pc}{\ensuremath{{\sf P}}}
\newcommand{\np}{\ensuremath{{\sf NP}}}
\newcommand{\npc}{\ensuremath{{\sf NPC}}}
\newcommand{\pspace}{\ensuremath{{\sf PSPACE}}}
\newcommand{\exptime}{\ensuremath{{\sf EXPTIME}}}
\newcommand{\CClassNP}{\textup{NP}\xspace}
\newcommand{\CClassP}{\textup{P}\xspace}

% Weitere pakete
\usepackage{multicol}
\usepackage{booktabs}

% Abbildungen
\usepackage{tikz}
\usetikzlibrary{arrows,calc}

% Meistens ist \varphi schöner als \phi, genauso bei \theta
\renewcommand{\phi}{\varphi}
\renewcommand{\theta}{\vartheta}

% Aufzählungen anpassen (alternativ: \arabic, \alph)
\renewcommand{\labelenumi}{(\roman{enumi})}

% rwth colors
% colors: blue violet purple carmine red magenta orange yellow grass cyan gold silver
\definecolor{rwth-blue}{cmyk}{1,.5,0,0}\colorlet{rwth-lblue}{rwth-blue!50}\colorlet{rwth-llblue}{rwth-blue!25}
\definecolor{rwth-violet}{cmyk}{.6,.6,0,0}\colorlet{rwth-lviolet}{rwth-violet!50}\colorlet{rwth-llviolet}{rwth-violet!25}
\definecolor{rwth-purple}{cmyk}{.7,1,.35,.15}\colorlet{rwth-lpurple}{rwth-purple!50}\colorlet{rwth-llpurple}{rwth-purple!25}
\definecolor{rwth-carmine}{cmyk}{.25,1,.7,.2}\colorlet{rwth-lcarmine}{rwth-carmine!50}\colorlet{rwth-llcarmine}{rwth-carmine!25}
\definecolor{rwth-red}{cmyk}{.15,1,1,0}\colorlet{rwth-lred}{rwth-red!50}\colorlet{rwth-llred}{rwth-red!25}
\definecolor{rwth-magenta}{cmyk}{0,1,.25,0}\colorlet{rwth-lmagenta}{rwth-magenta!50}\colorlet{rwth-llmagenta}{rwth-magenta!25}
\definecolor{rwth-orange}{cmyk}{0,.4,1,0}\colorlet{rwth-lorange}{rwth-orange!50}\colorlet{rwth-llorange}{rwth-orange!25}
\definecolor{rwth-yellow}{cmyk}{0,0,1,0}\colorlet{rwth-lyellow}{rwth-yellow!50}\colorlet{rwth-llyellow}{rwth-yellow!25}
\definecolor{rwth-grass}{cmyk}{.35,0,1,0}\colorlet{rwth-lgrass}{rwth-grass!50}\colorlet{rwth-llgrass}{rwth-grass!25}
\definecolor{rwth-green}{cmyk}{.7,0,1,0}\colorlet{rwth-lgreen}{rwth-green!50}\colorlet{rwth-llgreen}{rwth-green!25}
\definecolor{rwth-cyan}{cmyk}{1,0,.4,0}\colorlet{rwth-lcyan}{rwth-cyan!50}\colorlet{rwth-llcyan}{rwth-cyan!25}
\definecolor{rwth-teal}{cmyk}{1,.3,.5,.3}\colorlet{rwth-lteal}{rwth-teal!50}\colorlet{rwth-llteal}{rwth-teal!25}
\definecolor{rwth-gold}{cmyk}{.35,.46,.7,.35}
\definecolor{rwth-silver}{cmyk}{.39,.31,.32,.14}



% Header i-> inner (bei einseitig links), c -> center, o -> Outer (bei einseitg rechts)
\ihead{BuK WS 2020/21 \\ Tutorium 08 \\\today}
\chead{\Large Übungsblatt 1}
\ohead{Xiaoting Wang, 406267 \\
	   Wensheng Zhang, 405521 \\
	   Jiaming Yao, 416649}
	
\cfoot*{\pagemark} % Seitenzahlen unten

\begin{document}
	

	
	% Hier geht die eigentliche Lösung der Aufgaben los
	
	\section*{Aufgabe 4}
		$K_1$ ist eine Ja-Instanz. Die L"osung ist $3,8$, also  $\left[ \frac{bb}{b} \right]$, $\left[ \frac{aa}{baa} \right]$ ergeben oben und unten das gleich Wort $bbaa$.\\ \\
		$K_2$ ist eine Nein-Instanz. \\ 
		Begr"undung: \\
		Bei Startdomino gibt es zwei M"oglichkeiten: $\left[ \frac{ab}{abb} \right]$ oder $\left[ \frac{aa}{aab} \right]$ \\
		\underline{1.Fall:} Startdomino=$\left[ \frac{ab}{abb} \right]$, ben"otigt ein anderes Domino, das mit $b$ anf"angt und im oben\\
		\indent \indent \indent liegt, aber es gibt kein solches Domino in $K_2$.\\
		\underline{2.Fall:} Startdomino=$\left[ \frac{aa}{aab} \right]$, ben"otigt auch anderes Domino, das mit $b$ anf"angt und im\\
				\indent \indent \indent oben liegt, aber es gibt kein solches Domino in $K_2$.

		
	
		
	\section*{Aufgabe 5}
	Zu zeigen: eine Sprache $L$ ist endscheidbar $\iff$ $L$ auf die Sprache $L_{01}$ reduzierbar ist
	\textbf{$``\Rightarrow"$} zu zeigen: $L$ entscheidbar $\Rightarrow$ $L \le L_{01}$\\
	\indent \indent \ \ $\exists f $: $w \in L$ $\iff$ $f(w) \in L_{01}$\\
	\indent \indent \ \ F"ur rekursive Sprache $L$ existiert ein \textbf{TM} $M$, die $L$ entscheiden kann.\\ \\
	    \begin{tikzpicture}
	     \indent  \indent \indent  \draw[help lines] (-4,-2) grid (5,2);
	    
	    \draw[color=blue] (-2,-1) rectangle (2,1);
	    \draw[->] [color=blue]  (-3,0) -- (-2,0) ; 
	    \draw[->] [color=blue] (-2,0) -- (-1,0) ;
	      \draw[color=blue, fill=blue!35] (-1,-0.5) rectangle (1,0.5);
	      \draw[->] [color=blue]  (1,0.25) -- (2,0.25) ; 
	      \draw[->] [color=blue]  (1,-0.25) -- (2,-0.25) ; 
	      \draw[->] [color=blue]  (2,0.25) -- (3,0.25) ; 
	      \draw[->] [color=blue]  (2,-0.25) -- (3,-0.25) ; 
	    \node[color=blue] at (-2.5,0.25) {$w$};
	    \node[color=blue] at (0,0) {$M$};
	      \node[color=blue] at (1.5,0.5) {$1$};
	        \node[color=blue] at (1.5,-0.5) {$0$};
	          \node[color=blue] at (4,0.3) {$f(w)=01$};
	             \node[color=blue] at (4,-0.3) {$f(w)=10$};

	    \end{tikzpicture}
	    
	  \noindent  \textbf{Korrektheit:}\\
	   $w\in L$ $\Rightarrow$ $M$ akzeptiert $w$\\
	   \indent \indent  \indent $\Rightarrow$ $f(w)=01$\\
	     \indent \indent \indent $\Rightarrow$ $f(w)\in L_{01}$\\
	     
	\noindent      $w\notin L$ $\Rightarrow$ $M$ verwirft $w$\\
	   \indent \indent  \indent $\Rightarrow$ $f(w)=10$\\
	     \indent \indent \indent $\Rightarrow$ $f(w)\notin L_{01}$\\ \\
	     
	\noindent \textbf{ $``\Leftarrow"$} zu zeigen:  $L \le L_{01}$ $\Rightarrow$ $L$ entscheidbar  \\
	    \indent \indent \ \  \textit{Lemma:} Falls $L_1 \le L_2$ und $L_2$ rekursiv ist, ist $L_1$ auch rekursiv.\\
             \indent \indent \ \   zu zeigen: $L_{01}$ ist rekursiv\\
            \indent \indent \ \    F"ur $L_{01}$ konstruieren wir eine \textbf{2-Band-TM} $M_{01}$\\ \\
                \begin{tikzpicture}
                
              \indent  \indent \indent  \draw[help lines] (-4,-2) grid (5,3);
            \draw[violet]  (-3,1) -- (4,1);
            \draw[violet]  (-3,2) -- (4,2);
            
            \draw[orange] (-3,0) -- (4,0);
            \draw[orange] (-3,-1) -- (4,-1);
             \draw[violet] (-2,2) -- (-2,1);
              \draw[violet] (-1,2) -- (-1,1);
               \draw[violet] (0,2) -- (0,1);
                \draw[violet] (1,2) -- (1,1);
                 \draw[violet] (2,2) -- (2,1);
                  \draw[violet] (3,2) -- (3,1);
            
              \draw[orange] (-2,0) -- (-2,-1);
              \draw[orange] (-1,0) -- (-1,-1);
               \draw[orange] (0,0) -- (0,-1);
                \draw[orange] (1,0) -- (1,-1);
                 \draw[orange] (2,0) -- (2,-1);
                  \draw[orange] (3,0) -- (3,-1);
               \node[color=violet] at (-2.5,1.5) {$B$};
             \node[color=violet] at (-1.5,1.5) {$0$};
             \node[color=violet] at (-0.5,1.5) {$0$};
             \node[color=violet] at (0.5,1.5) {$1$};
              \node[color=violet] at (1.5,1.5) {$1$};
              \node[color=violet] at (2.5,1.5) {$B$};
              
            \node[color=orange] at(-2.5,-0.5) {$B$};
             \node[color=orange] at(-1.5,-0.5) {$0$};
              \node[color=orange] at(-0.5,-0.5) {$0$};
               \node[color=orange] at(0.5,-0.5) {$1$};
                \node[color=orange] at(1.5,-0.5) {$1$};
                 \node[color=orange] at(2.5,-0.5) {$B$};
                 
                 \draw[->][color=violet] (-1.5,0.25) -- (-1.5,1);
                  \draw[->][color=orange] (1.5,-1.75) -- (1.5,-1);
                  
                    \node[color=black] at(-3.5,2.5) {$M_{01}$};
            
            \end{tikzpicture}
            
               \begin{enumerate} 
               \item  Auf beide B"ander speichert $M$ die Eingabewort $w$.
               \item Lesek"opfe stehen auf dem ersten Zeichen auf 1.Band und auf dem letzten Zeichen auf 2.Band.
               \item Der 1.Kopf geht jeder Schritt nach \textbf{recht} und der 2.Kopf geht jeder Schritt nach \textbf{links}. Im jeden Schritt
                      soll $M$ ein 0 (vom 1.Band) und ein 1 (vom 2.Band) lesen, sonst wird $w$ verwirft.
               \item Dann l"auft $M_{01}$ wie 3.Schritt aber mit folgenden "Uberg"ange:
              
                \begin{tabular} {| l | c | r |}
                \hline
                1.Band & 2.Band & Aktion \\
                \hline
                0 & 1 & laufe weiter \\
                \hline
                1 & 0 & akzeptiert \\
                \hline
                0 & 0 & verwirft \\
                \hline
                1 & 1 & verwirft \\
                \hline
               
               \end{tabular}
               
               \end{enumerate}
	   
	\noindent   Mit obiger Konstruktion ist klar, dass $L_{01}$ durch $M_{01}$ entscheiden kann, d.h. $L_{01}$ ist entscheidbar.
	   Folglich ist $L$ entscheidbar (wegen des Lemmas).
	   
	
\section*{Aufgabe 6}
	Wir bezeichnen die Probleme von a) und b) als $PKP_a$ und $PKP_b$.
%\subsection*{a)}
%	$PKP_a$ ist entscheidbar.\\
%	\\ \ \\
%	Wir konstruieren einen Algorithmus, um $PKP_a$ zu entscheiden:
%	\\ \ \\
%	Sei $x_1,x_2,\cdots, x_k$ und $y_1,y_2,\cdots , y_k$ die Eingabe von $PKP_a$.\\
%	F"ur die Eingabe k"onnen wir eine kanonische Reihenfolge konstruieren, deren Elemente im Form von Menge ist, die(Menge) $1,\cdots,k$ enth"alt. Und die gleiche Zahl darf nicht 2-mal in der Menge vorkommen.
%	\begin{itemize}
%		\item [1)] $\{1\},\{2\},\cdots ,\{k\}$ \textcolor{blue}{(Jede Menge enth"alt ein Element und insgesamt gibt es ${k\choose 1}$ Mengen)}
%		\item [2)] $\{1,2\},\{1,3\},\cdots ,\{k-1,k\}$ \textcolor{blue}{(Jede Menge enth"alt zwei Elemente, und insgesamt gibt es ${k\choose 2}$ Mengen)}\\
%		\vdots
%		\item [k)] $\{1,2,\cdots,k\}$\textcolor{blue}{(Jede Menge enth"alt k-mal Elemente, und insgesamt gibt es ${k\choose k}$ Mengen)}
%	\end{itemize}
%	Eine solche kanonische Reihenfolge enth"alt ${k\choose 1}+{k\choose 2}+ \cdots +{k\choose k-1}+{k\choose k}$ mal Elemente(Menge), folglich ist es endlich und abz"ahlbar. \textcolor{blue}{(Eigentlich spielt die Reihenfolge hier keine Rolle. Wichtig ist, dass wir alle Kombinationen von $\{1,\cdots,k\}$ gefunden haben. Ist $PKP_a$ ein Ja-Instanz, muss seine L"osung in obiger Reihenfolge stehen.)}\\
%	Im n"achsten Schritt suchen wir eine L"osung f"ur $PKP_a$. Wir pr"ufen erst die Mengen nach der obiger Reihenfolge nacheinander in folgenden Maßen:
%	\begin{quote}
%		F"ur $i_1,i_2,\cdots,i_n \in {1,}$ aus der Menge z"ahlen wir, wie oft a und b jeweils in $x_{i_1},x_{i_2},\cdots, x_{i_n}$
%	\end{quote}
	
	
	
	
	
	
		  	
\subsection*{b)}
	Zu zeigen: $K\in PKP \iff f(K) \in PKP_b$
	\\ \ \\
	Die Beschreibung von f:\\
	Sei \textit{K} die die Eingabe f"ur PKP und $K =\Big\{ \left[ \frac{x_1}{y_1}  \right] ,\cdots , \left[ \frac{x_k}{y_k}  \right] \Big\} $.
\begin{itemize}
\item [Fall 1] Wenn es in \textit{K} keine Dominos gibt, wo das obere und das untere Wort gleich lang sind, so sei $f(K)=K$.
\item [Fall 2] Ansonsten machen wir f"ur die Dominos (oben und unten gleich lang) folgendes:\\
		Wir bezeichnen solche Dominos als $K'=\big\{  \left[ \frac{x_{l_{1}}}{y_{l_{1}}} \right], \cdots ,\left[ \frac{x_{l_{n}}}{y_{l_{n}}} \right] \big\}$ (die Reihenfolge spielt keine Rolle).\\
		Dann kombinieren wir Dominos aus $K'$ \textcolor{blue}{(oben und unten sind gleich lang)} mit einem anderen Domino aus $K\backslash K'$ \textcolor{blue}{ (oben und unten sind verschieden lang)}:\\
		Zum Beispiel k"onnen wir  $\left[ \frac{x_{l_{1}}}{y_{l_{1}}} \right]$ mit $\left[ \frac{x_i}{y_i}  \right]$ (einer aus $K \backslash K'$) kombinieren. In dem Fall bekommen wir zwei neue Dominos $\left[ \frac{x_{l_{1}} x_i}{y_{l_{1}} y_i} \right]$ und $\left[ \frac{x_i x_{l_{1}}}{y_i y_{l_{1}}} \right]$ \textcolor{blue}{ (einmal vorne, einmal hinter) }.
		\\ \ \\
		Insgesamt k"onnen wir $2 \times n \times (k-n)$ mal neue Dominos konstruieren. Man kann klar sehen, in solchen Dominos stehen immer oben und unten verschieden lang W"orter. Wir bezeichnen solche konstruierte Dominos als $K''$. So sei $f(K) = \{ K\backslash K', K''\} $.
	\end{itemize}  
	
	\noindent \textbf{Korrektheit:}
	
	\noindent \textcolor{blue}{$K \in PKP \Longrightarrow f(K) \in PKP_b$}\\
	Sei $(i_1,\cdots ,i_n)$ eine L"osung f"ur $K$, d.h.
	\begin{align*}
		x_{i_1} x_{i_2} \cdots x_{i_n} = y_{i_1} y_{i_2} \cdots y_{i_n}
	\end{align*}
	F"ur \textit{Fall 1} ist es klar, $f(K) \in PKP_b$.\\
	Im \textit{Fall 2} gibt es immer eine L"osung f"ur $f(K)$, dessen entsprechende W"orter gleich wie oben ist.\\
	Z.B. die L"ange von $x_{i_j}$ ist gleich wie $y_{i_j}$ f"ur ein $j \in \{1,\cdots, n\}$. So ein Domino steht nicht in $f(K)$. Aber k"onnen wir dieser durch $\left[ \frac{x_{i_{j-1}} x_{i_j}}{y_{i_{j-1}} y_{i_j}} \right]$ \textbf{oder} $\left[ \frac{x_{i_{j}} x_{i_{j+1}}}{y_{i_{j}} y_{i_{j+1}}} \right]$ aus $f(K)$ ersetzen und l"oschen die entsprechende Dominos:\\
	$$
	  \left[ \frac{x_{i_{1}}}{y_{i_{1}}} \right] \cdots  \left[ \frac{x_{i_{j}}}{y_{i_{j}}} \right]  \cdots \left[ \frac{x_{i_{n}}}{y_{i_{n}}} \right] \Longrightarrow \left[ \frac{x_{i_{1}}}{y_{i_{1}}} \right] \cdots  \left[ \frac{x_{i_{j}} x_{i_{j+1}}}{y_{i_{j}} y_{i_{j+1}}} \right] \left[ \frac{x_{i_{j+2}}}{y_{i_{j+2}}} \right] \cdots \left[ \frac{x_{i_{n}}}{y_{i_{n}}} \right]
	$$
	\textcolor{blue}{$K \notin PKP \Longrightarrow f(K) \notin PKP_b$}\\
	offensichtlich gilt wegen der Konstruktion \\
	\\ \ \\
	Wegen $K\in PKP \iff f(K) \in PKP_b$ und Unentscheidbarkeit von $PKP$ ist $PKP_b$ auch unentscheidbar.
	

	
	
	
	
	
	
	
	
	
	
	
	
	
	
	

	
\end{document}

