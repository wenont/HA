\documentclass[a4paper,11pt]{scrartcl}
\usepackage[a4paper, left=2cm, right=4.5cm, top=2cm, bottom=2cm]{geometry} % kleinere Ränder

%Paket für Header in Koma-Klassen (scrartcl, scrrprt, scrbook, scrlttr2)
\usepackage[headsepline]{scrlayer-scrpage}
% Header groß genug für 3 Zeilen machen
\setlength{\headheight}{3\baselineskip}


% Default Header löschen
\pagestyle{scrheadings}
\clearpairofpagestyles

% nicht kursiv gedruckte header
\setkomafont{pagehead}{\sffamily\upshape}

% Links im Dokuement sowie \url schön machen
\usepackage[colorlinks,pdfpagelabels,pdfstartview = FitH, bookmarksopen = true,bookmarksnumbered = true, linkcolor = black, plainpages = false, hypertexnames = false, citecolor = black]{hyperref}

% Umlaute in der Datei erlauben, auf deutsch umstellen
\usepackage[utf8]{inputenc}
\usepackage[ngerman]{babel}

% Mathesymbole und Ähnliches
\usepackage{amsmath}
\usepackage{mathtools}
\usepackage{amssymb}
\usepackage{microtype}


% Komplexitätsklassen
\newcommand{\pc}{\ensuremath{{\sf P}}}
\newcommand{\np}{\ensuremath{{\sf NP}}}
\newcommand{\npc}{\ensuremath{{\sf NPC}}}
\newcommand{\pspace}{\ensuremath{{\sf PSPACE}}}
\newcommand{\exptime}{\ensuremath{{\sf EXPTIME}}}
\newcommand{\CClassNP}{\textup{NP}\xspace}
\newcommand{\CClassP}{\textup{P}\xspace}

% Weitere pakete
\usepackage{multicol}
\usepackage{booktabs}

% Abbildungen
\usepackage{tikz}
\usetikzlibrary{arrows,calc}





% Meistens ist \varphi schöner als \phi, genauso bei \theta
\renewcommand{\phi}{\varphi}
\renewcommand{\theta}{\vartheta}

% Aufzählungen anpassen (alternativ: \arabic, \alph)
\renewcommand{\labelenumi}{(\roman{enumi})}


\usepackage{color}
\usepackage{pifont}

% rwth colors
% colors: blue violet purple carmine red magenta orange yellow grass cyan gold silver
\definecolor{rwth-blue}{cmyk}{1,.5,0,0}\colorlet{rwth-lblue}{rwth-blue!50}\colorlet{rwth-llblue}{rwth-blue!25}
\definecolor{rwth-violet}{cmyk}{.6,.6,0,0}\colorlet{rwth-lviolet}{rwth-violet!50}\colorlet{rwth-llviolet}{rwth-violet!25}
\definecolor{rwth-purple}{cmyk}{.7,1,.35,.15}\colorlet{rwth-lpurple}{rwth-purple!50}\colorlet{rwth-llpurple}{rwth-purple!25}
\definecolor{rwth-carmine}{cmyk}{.25,1,.7,.2}\colorlet{rwth-lcarmine}{rwth-carmine!50}\colorlet{rwth-llcarmine}{rwth-carmine!25}
\definecolor{rwth-red}{cmyk}{.15,1,1,0}\colorlet{rwth-lred}{rwth-red!50}\colorlet{rwth-llred}{rwth-red!25}
\definecolor{rwth-magenta}{cmyk}{0,1,.25,0}\colorlet{rwth-lmagenta}{rwth-magenta!50}\colorlet{rwth-llmagenta}{rwth-magenta!25}
\definecolor{rwth-orange}{cmyk}{0,.4,1,0}\colorlet{rwth-lorange}{rwth-orange!50}\colorlet{rwth-llorange}{rwth-orange!25}
\definecolor{rwth-yellow}{cmyk}{0,0,1,0}\colorlet{rwth-lyellow}{rwth-yellow!50}\colorlet{rwth-llyellow}{rwth-yellow!25}
\definecolor{rwth-grass}{cmyk}{.35,0,1,0}\colorlet{rwth-lgrass}{rwth-grass!50}\colorlet{rwth-llgrass}{rwth-grass!25}
\definecolor{rwth-green}{cmyk}{.7,0,1,0}\colorlet{rwth-lgreen}{rwth-green!50}\colorlet{rwth-llgreen}{rwth-green!25}
\definecolor{rwth-cyan}{cmyk}{1,0,.4,0}\colorlet{rwth-lcyan}{rwth-cyan!50}\colorlet{rwth-llcyan}{rwth-cyan!25}
\definecolor{rwth-teal}{cmyk}{1,.3,.5,.3}\colorlet{rwth-lteal}{rwth-teal!50}\colorlet{rwth-llteal}{rwth-teal!25}
\definecolor{rwth-gold}{cmyk}{.35,.46,.7,.35}
\definecolor{rwth-silver}{cmyk}{.39,.31,.32,.14}

% 请把新添加的宏包放置此处
\usepackage{blindtext} 
\usepackage{enumitem}
%\usepackage {ctex} % 中文字体支持
\usepackage[most]{tcolorbox} 
% https://tex.stackexchange.com/questions/180325/boxed-equation-with-number/180326
% http://mirrors.dotsrc.org/ctan/macros/latex/contrib/mathtools/empheq.pdf#page=23

\tcbset{colback=rwth-blue!10!white, colframe=-rwth-red!50!black, 
	highlight math style= {enhanced, %<-- needed for the ’remember’ options
		colframe=red,colback=red!10!white,boxsep=0pt}
} 
%   for tcolorbox
% 	Example
%	\begin{tcolorbox}[ams gather*]
%		\sum\limits_{n=1}^{\infty} \frac{1}{n} = \infty.\\
%		\int x^2 ~\text{d}x = \frac13 x^3 + c.
%	\end{tcolorbox}

% Codeblock and Code
\usepackage{listings}
\usepackage{xcolor}

\definecolor{codegreen}{rgb}{0,0.6,0}
\definecolor{codegray}{rgb}{0.5,0.5,0.5}
\definecolor{codepurple}{rgb}{0.58,0,0.82}
\definecolor{backcolour}{rgb}{0.95,0.95,0.92}

\lstdefinestyle{mystyle}{
	%	backgroundcolor=\color{backcolour},   
	backgroundcolor=\color{white},   
	commentstyle=\color{codegreen},
	keywordstyle=\color{magenta},
	numberstyle=\tiny\color{codegray},
	stringstyle=\color{codepurple},
	basicstyle=\ttfamily\footnotesize,
	breakatwhitespace=false,         
	breaklines=true,                 
	captionpos=b,                    
	keepspaces=true,                 
	numbers=left,                    
	numbersep=5pt,                  
	showspaces=false,                
	showstringspaces=false,
	showtabs=false,                  
	tabsize=2
}

\lstset{style=mystyle}
% https://upload.wikimedia.org/wikipedia/commons/2/2d/LaTeX.pdf#page=131

% Header i-> inner (bei einseitig links), c -> center, o -> Outer (bei einseitg rechts)
\ihead{BuK WS 2020/21 \\ Tutorium 08 \\\today}
\chead{\Large Übungsblatt 07}
\ohead{Jiaming Yao, 416649 \\
	   Xiaoting Wang, 406267 \\
	   Wensheng Zhang, 405521}	
\cfoot*{\pagemark} % Seitenzahlen unten

\begin{document}
	
\section*{Aufgabe 4}
\subsection*{a)}
\begin{itemize}
	\item[Zertifikat:] Sei $V'={v_1,v_2,\cdots ,v_k}$. Dann konstruieren wir das Zertifikat im Form $c=bin(v_1)\# bin(v_2)\# \cdots \# bin(v_{k-1})\# bin(v_k)\#$. Die L"ange des Zertifikat ist $O(n \log n)$ mit $n=|V|$, also poylnomiell in der Eingabel"ange.
	\item [Verifizierer:] 1. Pr"ufe zun"achst, ob die Eingabe die richtige Form hat.\\
	2. Pr"ufe dann, ob es f"ur alle Paare von Knoten $(v,w)\in V'\times V'$ keine Kante $e=(v,w)\in E$ existiert:\\
	- wir konstruieren eine Menge \textcolor{blue}{$M'$ von allen Paaren von Knoten $(v,w)\in V' \times V':\ M'=\{\{v_1,v_2\},\{v_1,v_3\},\cdots,\{v_{k-1},v_{k}\}\}$}. Schreibe $M'$ auf Arbeitsband.\\ (Manche kann benachbart sein, manche nicht. Unsere Ziel ist, solche benachbarte Knoten zu finden)\\
	- Iteriere aller paar von $M'$ und suche, ob es einen Kanten in dieser Paare gibt ( $(v,w)\in E$ ). Wenn ja, verwirft das Algorithmus. Wenn die Iterarion fertig ist, akzeptiert das Algorithmus.
	\item [Laufzeit:]
	\begin{itemize}
		\item Fomatcheck: linear
		\item Konstruiere $M'$. Es gibt maximal $(\frac{(k-1)\cdot k}{2})$ mal Paare in $M'$ mit $k=|V|$. So liegt die Laufzeit in $O(n^2)$. \\
		(z.B. in $V'$ gibt es vier Knoten. Dann hat die Menge $M'$ aus $V'$ 6-mal Paare, also (1,2),(1,3),(1,4),(2,3),(2,4),(3,4), 6 = 3+2+1. Es kann in der Formel $(\frac{(k-1)\cdot k}{2})$ repr"asentiert.)
		\item Bei Iteration kommt jede Paar (polynomiell viele, wie oben zeigt) von $M'$ genau einmal vor.\\
		F"ur jede Paar muss noch mal "uber Kantenliste gelaufen werden ( diese ist polynomiell lang).
		\item Insgesamt in Polynomialzeit.
	\end{itemize}


	\item[Korrektheit:] Wenn Verifizierer akzeptiert, dann l"asst isch aus dem Zertifikat $V'$ konsturieren und die Instanz ist k-Independent-Set.\\
	Wenn die Instanz k-Independent-Set ist, l"asst sich daraus ein g"ultiges Zertifikat konstruieren und der Verifizierer akzeptiert.
	
 
\end{itemize}
\subsection*{b)}
\begin{lstlisting} [language=java]
1. K:={1,...,N}   // N=|V|
2. IF A(K) = FALSE THEN Ausgabe NotFound 
3. FOR i:=1 to N do
       IF A(K\{i})=TRUE && |K|>1
       THEN K:=K\{i}
   ENDFOR
4. Ausgabe K	
\end{lstlisting}
Die Laufzeit des Algorithmus ist in der Polynomialzeit beschr"ankt, weil FOR nut N-mal ausgef"uht wird.
\\ \ \\
Korrektheit:\\
$G(=V,E)\in L$ $\Longrightarrow$ Der Algorithmus kann eine zul"assige L"osung $K\in V$ ausgeben.\\
$\Longrightarrow$ $G$ ist k-Independent Set, $K\in V$\\
$\Longrightarrow$ $\forall (v,w)\in K\ :\ (v,w)\notin E$
\\ \ \\
Warum ist ausgegebenes $K$ eine zul"assige L"osung?
\begin{itemize}
	\item Wir nehmen an, dass $\exists (v,w)\in K\ :\ (v,w)\in E$ gelte.
	\item D.h. es gibt mindesten einen Knoten, die mit anderen Knoten aus $K$ benachbart ist.
		\begin{itemize}
			\item[1.] $\exists i \in K$, so dass $A(K\{i\})=TRUE$.\\
			So ist es im Widerspruch dazu, dass der Algorithmus den Knoten i aus $K$ gestrichen hat.
			\item[2.] $\forall i \in K$, so dann $A(K\{i\})=FALSE$.\\
			Der Fall bedeutet, dass G kein k-Independent Set ist. So soll der Algorithmus kein $K$ sondern ein NotFound am Anfang ausgeben. Somit liegt es auch im Widerspruch.
		\end{itemize}
	\item Also gilt die Annahme nicht sondern ist das Aussage $\forall (v,w)\in K\ :\ (v,w)\notin E$ richtig.
	\item Somit ist ausgegebenes $K$ eine zul"assige L"osung.
\end{itemize}

\section*{Aufgabe 5}
\subsection*{a)}
\noindent \textcolor{blue}{Linker Graph}: Ja. Knoten\{1,2,5,7\} mit Farbe 1 und Knoten\{3,4,6\} mit Farbe 2.\\
\textcolor{blue}{Rechter Graph}: Nein. Denn falls Knoten 1 mit Farbe 1 gefärbt wird, dann muss Knoten 3 mit Farbe 2, Knoten 5 mit Farbe 1, Knoten 4 mit Farbe 2 und Knoten 7 mit Farbe 1, so dass $c(1)\ne c(7)$ nicht gilt.

\subsection*{b)}
\begin{itemize}
	\item Wir lösen das Problem mit einer Tiefensuche.
	\item In den Tiefensuche-Prozess färben wir alle Konten, indem jeder Knoten und seine Kinder verschiedene Farbe haben.
	\item Nachdem Tiefensuche-Prozess, suchen wir die Kanten, die nicht in der Tiefensuche sind. Vergleichen wir dann, ob jede Kante zwei verschiedenen gefärbten Knoten hat.\\
	Falls \textcolor{blue}{ja} $\Rightarrow$ Der Graph gilt 2-COLORABILITY.
	\item Die Laufzeit  von diesem Algorithmus ist \textcolor{blue}{polynomiell beschränkt} (nach der Vorlesung).
\end{itemize}

\subsection*{c)}
\noindent Nach der Vorlesung ist schon erkennt, dass COLORING $\leq_p$ SAT gilt. Deswegen hat COLORING einen Polynomialzeitalgorithmus.\\
Bei diesem Fall ist 3-COLORABILITY ein spezieller Fall von COLORING, so dass 3-COLORABILITY $\leq_p$ SAT. Weil SAT eine NP-Problem ist, ist 3-COLORABILITY auch ein NP-Problem.


\end{document}

