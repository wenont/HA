\documentclass[a4paper,11pt]{scrartcl}
\usepackage[a4paper, left=2cm, right=4.5cm, top=2cm, bottom=2cm]{geometry} % kleinere Ränder

%Paket für Header in Koma-Klassen (scrartcl, scrrprt, scrbook, scrlttr2)
\usepackage[headsepline]{scrlayer-scrpage}
% Header groß genug für 3 Zeilen machen
\setlength{\headheight}{3\baselineskip}


% Default Header löschen
\pagestyle{scrheadings}
\clearpairofpagestyles

% nicht kursiv gedruckte header
\setkomafont{pagehead}{\sffamily\upshape}

% Links im Dokuement sowie \url schön machen
\usepackage[colorlinks,pdfpagelabels,pdfstartview = FitH, bookmarksopen = true,bookmarksnumbered = true, linkcolor = black, plainpages = false, hypertexnames = false, citecolor = black]{hyperref}

% Umlaute in der Datei erlauben, auf deutsch umstellen
\usepackage[utf8]{inputenc}
\usepackage[ngerman]{babel}

% Mathesymbole und Ähnliches
\usepackage{amsmath}
\usepackage{mathtools}
\usepackage{amssymb}
\usepackage{microtype}


% Komplexitätsklassen
\newcommand{\pc}{\ensuremath{{\sf P}}}
\newcommand{\np}{\ensuremath{{\sf NP}}}
\newcommand{\npc}{\ensuremath{{\sf NPC}}}
\newcommand{\pspace}{\ensuremath{{\sf PSPACE}}}
\newcommand{\exptime}{\ensuremath{{\sf EXPTIME}}}
\newcommand{\CClassNP}{\textup{NP}\xspace}
\newcommand{\CClassP}{\textup{P}\xspace}

% Weitere pakete
\usepackage{multicol}
\usepackage{booktabs}

% Abbildungen
\usepackage{tikz}
\usetikzlibrary{arrows,calc}





% Meistens ist \varphi schöner als \phi, genauso bei \theta
\renewcommand{\phi}{\varphi}
\renewcommand{\theta}{\vartheta}

% Aufzählungen anpassen (alternativ: \arabic, \alph)
\renewcommand{\labelenumi}{(\roman{enumi})}


\usepackage{color}
\usepackage{pifont}

% rwth colors
% colors: blue violet purple carmine red magenta orange yellow grass cyan gold silver
\definecolor{rwth-blue}{cmyk}{1,.5,0,0}\colorlet{rwth-lblue}{rwth-blue!50}\colorlet{rwth-llblue}{rwth-blue!25}
\definecolor{rwth-violet}{cmyk}{.6,.6,0,0}\colorlet{rwth-lviolet}{rwth-violet!50}\colorlet{rwth-llviolet}{rwth-violet!25}
\definecolor{rwth-purple}{cmyk}{.7,1,.35,.15}\colorlet{rwth-lpurple}{rwth-purple!50}\colorlet{rwth-llpurple}{rwth-purple!25}
\definecolor{rwth-carmine}{cmyk}{.25,1,.7,.2}\colorlet{rwth-lcarmine}{rwth-carmine!50}\colorlet{rwth-llcarmine}{rwth-carmine!25}
\definecolor{rwth-red}{cmyk}{.15,1,1,0}\colorlet{rwth-lred}{rwth-red!50}\colorlet{rwth-llred}{rwth-red!25}
\definecolor{rwth-magenta}{cmyk}{0,1,.25,0}\colorlet{rwth-lmagenta}{rwth-magenta!50}\colorlet{rwth-llmagenta}{rwth-magenta!25}
\definecolor{rwth-orange}{cmyk}{0,.4,1,0}\colorlet{rwth-lorange}{rwth-orange!50}\colorlet{rwth-llorange}{rwth-orange!25}
\definecolor{rwth-yellow}{cmyk}{0,0,1,0}\colorlet{rwth-lyellow}{rwth-yellow!50}\colorlet{rwth-llyellow}{rwth-yellow!25}
\definecolor{rwth-grass}{cmyk}{.35,0,1,0}\colorlet{rwth-lgrass}{rwth-grass!50}\colorlet{rwth-llgrass}{rwth-grass!25}
\definecolor{rwth-green}{cmyk}{.7,0,1,0}\colorlet{rwth-lgreen}{rwth-green!50}\colorlet{rwth-llgreen}{rwth-green!25}
\definecolor{rwth-cyan}{cmyk}{1,0,.4,0}\colorlet{rwth-lcyan}{rwth-cyan!50}\colorlet{rwth-llcyan}{rwth-cyan!25}
\definecolor{rwth-teal}{cmyk}{1,.3,.5,.3}\colorlet{rwth-lteal}{rwth-teal!50}\colorlet{rwth-llteal}{rwth-teal!25}
\definecolor{rwth-gold}{cmyk}{.35,.46,.7,.35}
\definecolor{rwth-silver}{cmyk}{.39,.31,.32,.14}

% 请把新添加的宏包放置此处
\usepackage{blindtext} 
\usepackage{enumitem}
%\usepackage {ctex} % 中文字体支持
\usepackage[most]{tcolorbox} 
% https://tex.stackexchange.com/questions/180325/boxed-equation-with-number/180326
% http://mirrors.dotsrc.org/ctan/macros/latex/contrib/mathtools/empheq.pdf#page=23

\tcbset{colback=rwth-blue!10!white, colframe=-rwth-red!50!black, 
	highlight math style= {enhanced, %<-- needed for the ’remember’ options
		colframe=red,colback=red!10!white,boxsep=0pt}
} 
%   for tcolorbox
% 	Example
%	\begin{tcolorbox}[ams gather*]
%		\sum\limits_{n=1}^{\infty} \frac{1}{n} = \infty.\\
%		\int x^2 ~\text{d}x = \frac13 x^3 + c.
%	\end{tcolorbox}

% Codeblock and Code
\usepackage{listings}
\usepackage{xcolor}

\definecolor{codegreen}{rgb}{0,0.6,0}
\definecolor{codegray}{rgb}{0.5,0.5,0.5}
\definecolor{codepurple}{rgb}{0.58,0,0.82}
\definecolor{backcolour}{rgb}{0.95,0.95,0.92}

\lstdefinestyle{mystyle}{
	%	backgroundcolor=\color{backcolour},   
	backgroundcolor=\color{white},   
	commentstyle=\color{codegreen},
	keywordstyle=\color{magenta},
	numberstyle=\tiny\color{codegray},
	stringstyle=\color{codepurple},
	basicstyle=\ttfamily\footnotesize,
	breakatwhitespace=false,         
	breaklines=true,                 
	captionpos=b,                    
	keepspaces=true,                 
	numbers=left,                    
	numbersep=5pt,                  
	showspaces=false,                
	showstringspaces=false,
	showtabs=false,                  
	tabsize=2
}

\lstset{style=mystyle}
% https://upload.wikimedia.org/wikipedia/commons/2/2d/LaTeX.pdf#page=131

% Header i-> inner (bei einseitig links), c -> center, o -> Outer (bei einseitg rechts)
\ihead{BuK WS 2020/21 \\ Tutorium 08 \\\today}
\chead{\Large Weihnachtblatt}
\ohead{Jiaming Yao, 416649 \\
	   Xiaoting Wang, 406267 \\
	   Wensheng Zhang, 405521}	
\cfoot*{\pagemark} % Seitenzahlen unten

\begin{document}
	
\section*{Aufgabe 1}

L enth"alt genau ein Wort s. s ist entweder 0 oder 1. \\
Dass L entscheidbar ist, bedeutet, es gibt eine TM, die auf jeder Eingabe h"alt und jedes Word aus L($\Sigma ^* \backslash $ L) akzeptiert(verwirft).\\
Aber wir wissen nicht, welches Wort genau in der Sprache steht(0 oder 1?). Wir k"onnen nicht so eine TM entwerfen, die eine unbekannte Sprache entscheiden kann. Deshalb ist L unentscheidbar.

\section*{Aufgabe 2}

F"ur die unendliche Rezepte $w_i$ und $M_i$ definieren wir eine zweidimensionale unendliche Matrix ($A_{m,n}$)$_{m\in \mathbb{N},n\in \mathbb{N}}$ mit
$$ A_{m,n}=\left\{
\begin{aligned}
1 &\ falls\ M_m\ w_n\ akzeptiert,\\
0 &\ sonst
\end{aligned}
\right.
$$
Beispiel:
\begin{table}[h]
	\begin{tabular}{l|lllll}
		& $w_0$ & $w_1$   & $w_2$   & $w_3$   & $\cdots$ \\ \hline
		$M_0$   & 0       & 1       & 0       & 1       & $\cdots$ \\
		$M_1$   & 1       & 0       & 1       & 0       & $\cdots$ \\
		$M_2$   & 0       & 0       & 1       & 1       & $\cdots$ \\
		$M_3$   & 1       & 0       & 1       & 1       & $\cdots$ \\
		$\vdots$ & $\vdots$ & $\vdots$ & $\vdots$ & $\vdots$ & $\ddots$ 
	\end{tabular}
\end{table}

Die Matrix $A_{m,n}$ passt unsere Aufgabe: z.B. $M_0$ akzeptiert $w_1$ und $w_3$, verwirft $w_0$ und $w_2$. D.h. der 0-te Wichtel mag die Leckereien, die zu den 1-ten und 3-ten Rezepten geh"oren. \\
Sei $D = \{ w_m\ |\ A_{m,m} = 0\}$,\\
Also k"onnen wir $D$ als Diagonalsprache betrachten. Aus Vorlesung sind $D$ und $\overline{D}$ nicht rekursiv.\\
Somit ist das in der Aufgabe genannte Problem unentschiedbar.

\section*{Aufgabe 3}

\subsection*{a)}

Wir k"onnen eine TM $M_a$ konstruieren, die $L_1$ entscheidet:\\
\begin{itemize}
	\item Zuerst "uberpr"uft $M_a$, ob die Eingabe syntaktisch (als G"odelnummer) korrekt ist.
	\item Dann z"ahle, wie viele Zust"ande in der Eingabe bzw. G"odelnummer auftreten. Z.B. \\
	111 \textcolor{red}{$\underline{0}$101010001000} 11 \textcolor{red}{$\underline{0}$100100010001000} 11 \textcolor{red}{$\underline{0}$10001001000100} 11
\\
	\textcolor{red}{$\underline{000}$10101001000} 11 \textcolor{red}{$\underline{000}$100100101000} 11 \textcolor{red}{$\underline{000}$100010100010} 111 \\ 
	$M_a$ kann z"ahlen, wie viele 0s es unmittelbar nach jedes 11 gibt (wie Unterstreichen zeigt). Dann kann $M_a$ die Information auf zweitem Band speichern.
	\item Wenn das Z"ahlen fertig ist, pr"uft, ob $M_a$ die Anzahl von die Zust"anden mehr als 24 ist. Wenn ja, akzeptiert die Eingabe, sonst verwirft.
\end{itemize}
Wenn das Endstand nicht gez"ahlt ist, kann die Anzahl von die Zust"ande automatisch eins addieren. 



\end{document}

