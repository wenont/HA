\documentclass[a4paper,11pt]{scrartcl}
\usepackage[a4paper, left=2cm, right=4.5cm, top=2cm, bottom=2cm]{geometry} % kleinere Ränder

%Paket für Header in Koma-Klassen (scrartcl, scrrprt, scrbook, scrlttr2)
\usepackage[headsepline]{scrlayer-scrpage}
% Header groß genug für 3 Zeilen machen
\setlength{\headheight}{3\baselineskip}


% Default Header löschen
\pagestyle{scrheadings}
\clearpairofpagestyles

% nicht kursiv gedruckte header
\setkomafont{pagehead}{\sffamily\upshape}

% Links im Dokuement sowie \url schön machen
\usepackage[colorlinks,pdfpagelabels,pdfstartview = FitH, bookmarksopen = true,bookmarksnumbered = true, linkcolor = black, plainpages = false, hypertexnames = false, citecolor = black]{hyperref}

% Mathesymbole und Ähnliches
\usepackage{amsmath}
\usepackage{mathtools}
\usepackage{amssymb}
\usepackage{microtype}
\newcommand{\NN}{{\mathbb N}}
\newcommand{\RR}{{\mathbb R}}
\newcommand{\QQ}{{\mathbb Q}}
\newcommand{\ZZ}{{\mathbb{Z}}}

% Meistens ist \varphi schöner als \phi, genauso bei \theta
\renewcommand{\phi}{\varphi}
\renewcommand{\theta}{\vartheta}

% Header i-> inner (bei einseitig links), c -> center, o -> Outer (bei einseitg rechts)
\ihead{AI WS 2020/21 \\ Tutorium 168 \\\today}
\chead{\Large Assignment 01}
\ohead{2, 1 \\
	1, 2 \\
	Wensheng Zhang, 405521}	
\cfoot*{\pagemark} % Seitenzahlen unten

% Codeblock and Code
\usepackage{listings}
\usepackage{xcolor}

\definecolor{codegreen}{rgb}{0,0.6,0}
\definecolor{codegray}{rgb}{0.5,0.5,0.5}
\definecolor{codepurple}{rgb}{0.58,0,0.82}
\definecolor{backcolour}{rgb}{0.95,0.95,0.92}

\lstdefinestyle{mystyle}{
	backgroundcolor=\color{backcolour},   
	commentstyle=\color{codegreen},
	keywordstyle=\color{magenta},
	numberstyle=\tiny\color{codegray},
	stringstyle=\color{codepurple},
	basicstyle=\ttfamily\footnotesize,
	breakatwhitespace=false,         
	breaklines=true,                 
	captionpos=b,                    
	keepspaces=true,                 
	numbers=left,                    
	numbersep=5pt,                  
	showspaces=false,                
	showstringspaces=false,
	showtabs=false,                  
	tabsize=2
}

\lstset{style=mystyle}


\begin{document}


\section*{Exercise 1.1}

\subsection*{(a)}

	Yes. The elevator can perceive if each button on floors and in the elevator ON or OFF. On the basis of that perception the elevator makes a set of actions, aiming at the right goal. And both of perception and action are indicated the knowledge of the world, which are represented by the states of the elevator and the buttons.\\
		
	\begin{description}
		\item[percepts] state of every buttons, current load of elevator, obstacle in the door
		\item[actions] move elevator, open and close door, turn ON or OFF the buttons, check moveableness, alarm
		\item[goals] take object from floor A to floor B
		\item[environment] elevator
	\end{description}
	
	The elevator is reflexive agent with states.
	
\subsection*{(b)}
	
	\begin{itemize}
		\item	how fast reach the elevator the aimed floor
		\item	comfortableness of people
		\item	stability of the elevator
		\item	barrier-free
		\item	noise emission
		\item	security
	\end{itemize}

\subsection*{(c)}


\section*{Exercise 1.2}

	\begin{table}[h!]
		\begin{tabular}{llllll}
			& accessible & deterministic & episodic & dynamic & discrete \\
			elevator     & Y          & N             & N        & Y       & Y        \\
			the internet & N          & N             & N        & Y       & Y        \\
			Mars Rover   & N          & N             & Y        & Y       & N       
		\end{tabular}
	\end{table}


\section*{Exercise 1.3}

\subsection*{(a)}

	\begin{lstlisting}[language=java]
		loop{
			if(IsDirty()) Suck;
			while(IsEmpty(left)){
				Move left;
				if(IsDirty()) Suck;
			}
		
			if(IsEmpty(up)){
				Move up;
				if(IsDirty()) Suck;
			}else{
				endloop
			}
		
			while(IsEmpty(right)){
				Move right;
				if(IsDirty()) Suck;
			}
		
			if(IsEmpty(up)){
				Move up;
				if(IsDirty()) Suck;
			}else{
				endloop
			}
		}

	\end{lstlisting}

	
	

	
	


\end{document}
